A\+PI for Circular buffer which can holds any type of object.

1.\+First step in using this A\+PI is to initiate circular buffer using init\+\_\+\+CB, which will return status massage if CB is successfully created or not. 2.\+Another two functions of importance are \mbox{\hyperlink{circular__buffer_8h_a0cd402749c474cfcd32b155e2e10ffd3}{C\+B\+\_\+put()}} and \mbox{\hyperlink{circular__buffer_8h_ada7f8b98a82080f6184b70bc1bd82b91}{C\+B\+\_\+get()}} function, which are used for pushing item in buffer and reading item from it. 3.\+At the beginning the pointer value of field head\+Ptr and tail\+Ptr will be same, as the items are inserted and read over time, the pointers will increase. 4.\+If the buffer is full and there are more items to push in it, the oldest items will be overwritten. 